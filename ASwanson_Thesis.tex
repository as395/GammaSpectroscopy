
%% Preamble
\documentclass[12pt,twoside]{reedthesis}

\usepackage{graphicx,latexsym} 
\usepackage{amssymb,amsthm,amsmath}
\usepackage{longtable,booktabs,setspace} 
\usepackage{chemarr} 
\usepackage[hyphens]{url}
\usepackage{rotating}
\usepackage{natbib}
% Comment out the natbib line above and uncomment the following two lines to use the new 
% biblatex-chicago style, for Chicago A. Also make some changes at the end where the 
% bibliography is included. 
%\usepackage{biblatex-chicago}
%\bibliography{thesis}

\usepackage{times} 

\title{Application of a correction scheme for coincidence summing in $\gamma$-ray spectroscopy}
\author{Amanda Elise Swanson}
% The month and year that you submit your FINAL draft TO THE LIBRARY (May or December)
\date{May 2018}
\division{Mathematics and Natural Sciences}
\advisor{Jenna K. Smith}
%If you have two advisors for some reason, you can use the following
\altadvisor{Daniel Gerrity}
%%% Remember to use the correct department!
\department{The Established Interdisciplinary Committee for Physics and Chemistry}
% if you're writing a thesis in an interdisciplinary major,
% uncomment the line below and change the text as appropriate.
% check the Senior Handbook if unsure.
\thedivisionof{}
% if you want the approval page to say "Approved for the Committee",
% uncomment the next line
%\approvedforthe{Committee}

\setlength{\parskip}{0pt}
%%
%% End Preamble
%%
%% The fun begins:
\begin{document}

  \maketitle
  \frontmatter % this stuff will be roman-numbered
  \pagestyle{empty} % this removes page numbers from the frontmatter

    \chapter*{Acknowledgements}
	I want to thank a few people. Many people, in fact!

%    \chapter*{Preface}
    
    \chapter*{List of Abbreviations}

	\begin{table}[h]
	\centering % You could remove this to move table to the left
	\begin{tabular}{ll}
		\textbf{ABC}  	&  American Broadcasting Company \\
		\textbf{CBS}  	&  Columbia Broadcasting System\\
		\textbf{CDC}  	&  Center for Disease Control \\
		\textbf{CIA}  	&  Central Intelligence Agency\\
		\textbf{CLBR} 	&  Center for Life Beyond Reed\\
		\textbf{CUS}  	&  Computer User Services\\
		\textbf{FBI}  	&  Federal Bureau of Investigation\\
		\textbf{NBC}  	&  National Broadcasting Corporation\\
	\end{tabular}
	\end{table}
	

%    \tableofcontents
% if you want a list of tables, optional
%    \listoftables
% if you want a list of figures, also optional
%    \listoffigures

% If your abstract is longer than a page, there may be a formatting issue.
    \chapter*{Abstract}
        T. M. Semkow's method for correcting peak intensity ratios in gamma ray spectroscopy due to summing in detectors was applied to a python code ...	
%	\chapter*{Dedication}

  \mainmatter % here the regular arabic numbering starts
  \pagestyle{fancyplain} % turns page numbering back on

    \chapter*{Introduction}
         \addcontentsline{toc}{chapter}{Introduction}
	\chaptermark{Introduction}
	\markboth{Introduction}{Introduction}
	% The three lines above are to make sure that the headers are right, that the intro gets included in the table of contents, and that it doesn't get numbered 1 so that chapter one is 1.

% \singlespacing
% \onehalfspacing
% \doublespacing
    \section{Nuclear Decay and Radiation Detection}
Introduce the significance and use of nuclear decay!
\subsection{Isotope Terminology}
     \begin{equation}
     \centering
     \mathrm{^{A}_{Z}X_{N}}
     \end{equation}
     
    $A$ refers to nucleons, $Z$ refers to protons, and $N$ refers to neutron number. $N$ is frequently omitted for the sake of clarity, since $N = Z - A$.
\subsection{Types of decay}
  Charged particle radiation (i.e. heavy ions or electrons) interact iwth matter primarily via coulomb interactions, whereas neutral particle radiation interacts with the nuclei of atoms, often creating secondary charged particles. \cite{Knoll} Since this thesis focuses of gamma radiation, I will further describe how gamma radiation affects matter later on.
  	
\subsection{Decay schemes}
 As shown in figure \ref{decay}, the information needed to construct a decay scheme for a radioactive substance includes the feeding fractions, which dictate the ratio of gamma rays initially deposited to each energy level, and the branching ratios to determine the most likely pathways for a gamma ray to decay. The decay scheme...
    	\begin{figure}[h!]
	       \centering
	    \includegraphics{decayscheme_semkow.png}
	     \caption{A typical decay scheme including the feeding fractions ($f$), the energy levels ($n$), and the possible decay branches ($x$). Image taken from Semkow.}
	 \label{decay}
	\end{figure}

  
\subsection{Gamma radiation - interactions with matter}
 Gamma radiation interacts with matter in three significant ways: by photoelectric absorption, Compton scattering, and pair producion \cite{Knoll}. These interactions dominate at low, mid, and high energy ranges, respectively \cite{Gilmore}.

Photoelectric:  As shown in figure \ref{photo}, the photon is completely absorbed and a an energetic ``photoelectron" is ejected from the innermost shell of an atom. Following this ejection, an electrons from a higher valence shell emits a photon when relaxing to the lower unoccupied shell.
  \begin{equation}
     E_{e-} = E_{\gamma} - E_{b}
     \label{photoelec}
     \end{equation}
 $E_{b}$ is the binding energy of a photoelectron.
 \begin{figure}[h!]
	       \centering
	    % DO NOT ADD A FILENAME EXTENSION TO THE GRAPHIC FILE
	    \includegraphics{photoelectric.png}
	     \caption{Photoelectric effect. Image taken from Gilmore.}
	 \label{photo}
	\end{figure}
 
Compton: Via direct interaction with an electron, the gamma ray loses part of its energy. In extreme cases, the gamma ray can lose all of its energy (full recoil), or a small part of its energy (grazing scatter), but it can never retain its full energy at the end of the interaction. Figure \ref{compt} shows the mechanism of Compton scattering:
 \begin{figure}[h!]
	       \centering
	    % DO NOT ADD A FILENAME EXTENSION TO THE GRAPHIC FILE
	    \includegraphics{comptonscatter.png}
	     \caption{Compton Scattering mechanism. Image taken from Gilmore.}
	 \label{compt}
	\end{figure}
	  \begin{equation}
     E_{\gamma} = E_{e} - E_{\gamma}\left(\frac{1}{1+ \frac{E_{\gamma}(1-cos\theta)}{mc^{2}}}\right)
     \label{photoelec}
     \end{equation}
	It is simplest to assume that the electrons in these interactions are not tightly bound to a nucleus, because most will be in the outer valence shells and will be fairly easy to interact with. Probability of occurrance increases linearly with Z, since for higher-charge atoms there are more electrons to interact with.
	
Pair production: conversion of gamma-ray to electron/positron pair. Direct interaction with the nucleus. This can only occur if the incident $\gamma$-ray exceeds twice the energy of a rest-mass electron.

\subsection{Attenuation coefficients}
``Probability of occurrance per unit path length of the absorber" \cite{Knoll}.
    \begin{equation}
     \mu = \tau + \sigma + \kappa
     \label{attenuation}
     \end{equation}

\subsection{Radiation Detection}
	Since gamma-rays are not charged, it is necessary to force them to interact with material to create ``fast electrons", which provide a useful signal with measureable energies. Ideally, a detector would be consistently capable of converting a gamma ray's full energy into an electron, but as discussed above, there is significant energy loss in the particle interactions. There is also further energy loss of the fast electrons via bremsstrahlung and ionization within the absorber material.
	
	Two requirements: the detector must 1) have a high probability of $\gamma$ interactions and 2) must be capable of effectively detecting the secondary electrons. It should also be large enough to prevent loss of secondary electrons scattered at large angles, and thick enough to ensure that most interactions are not scattered out of the detector.
\subsection{Quantum Mechanical Observations (?)}
  Uncertainty principle, spin, parity, static vs. dynamic properties
  
  Need to fill out more here!
  
\chapter{Measuring the Impact of Coincidence Summing}
 
\section{Parameters Affecting Coincidence Summing}
Summing in and/or out is a result of multiple gamma emissions hitting a detector at nearly the same time. The lifetimes of energy states populated by $\beta^{-}$ decay have lifetimes that are shorter than the time needed for a detector to distinguish two separate emissions (timescale?). 

In the simplest case, two coincident $\gamma$-rays hit a detector at once. In this case, the summed energy delivered to the detector corresponds to a reduction of measured counts for each photopeak, or summing out of both peaks. If the value of the summed energies matches the energy of another photopeak in the spectrum, as shown in Figure \ref{sum_in}, this summing also contributes to an artificially inflated measure for the larger photopeak, or summing in.

 \begin{figure}[h!]
	\centering
	\includegraphics[width=12cm]{summingin.png}
	\caption{Simple demonstration of how peak areas for gamma rays of different energies are affected by coincidence summing. a) Measured peak areas without summing, b) Measured peaks with coincidence summing; counts of E$_{1}$ and E$_{2}$ are reduced, and the counts for E$_{3}$ are increased.}
\label{sum_in}
\end{figure}

Summing out occurs more frequently than summing in, so it is expected that summing would produce fewer total photopeak counts over the spectrum. This is because summing in requires a simultaneous detection of two full energy gamma-rays to increase the counts of a gamma ray with higher energy, whereas a gamma-ray can sum out if it is detected at the same time as any other additional energy, such as an emitted gamma-ray that has undergone Compton scattering or Bremsstrahlung.

The degree of summing depends on parameters of the source producing gamma rays as well as the detector; since the summing is dependent on the probability that two gamma-rays will be detected in a crystal at the same time, and the direction of their emissions are random, the subtended angle of the detector with respect to the source should be minimized to reduce the amount of summing. As shown in figure \ref{angle}, detectors positioned nearer to the source (with ``close geometries") will experience a larger number of coincident gamma rays than those farther away, and increasing the number of individual detectors for a surface area will also reduce the probability of simultaneous detection of two gamma-rays without sacrificing the total number of counts.

 \begin{figure}[h!]
	\centering
	\includegraphics[width=12cm]{detector_angles.png}
	\caption{Simple demonstration of the impact of detector geometry on the degree of coincidence summing. Individual detectors are shown as grey panels, and the source is represented as a ball. a) A large detector close to the source. b) A large detector farther away from the source. c) Three small detectors far away from the source.}
\label{angle}
\end{figure}

A summary of parameters affecting the degree of true coincidence summing for an experiment is described in G. Gilmore's textbook on \textit{Practical Gamma Ray Spectroscopy}:

``- [Summing] usually results in lower full-energy peak areas," since summing out does not necessarily produce summing in,

``- it gets worse the closer the source is to the detector," since a wider angle of separation between simulnateously emitted gamma rays can still result in coincident detection,

``- it gets worse the larger the detector (and is worst of all when using a well detector)," since large detector surface areas increase the number of coincident detections,

``- it may be worse if a detector with a thin window is used because the X-rays that contribute to the summing will not be absorbed," and if low energy gamma rays are not absorbed before detection, they will contribute to more summing out,

``- it can be expected whenever nuclides with a complex decay scheme are measured," since a greater number of cascading gamma emissions will result in more summing, and

``- The degree of summing is not dependent upon count rate." \cite{Gilmore}

\section{Mathematical Corrections}

\section{Semkow's Matrix Method}

Semkow's method is particularly useful to describe the impact of 

\begin{align}
c_{ji} &= \frac{x_{ji}}{1+\alpha_{ji}},\\
a_{ji} &= c_{ji}\epsilon^{p}_{ji},\\
e_{ji} &= c_{ji}\epsilon^{t}_{ji},\\
b_{ji} &= x_{ji}-e_{ji},
\label{Sdefinitions}
\end{align}

\section{Simple Coincidence Summing}

To determine the degree of coincidence summing for a sample, it is useful to develop a mathematical understanding of how summing in and out affects the probability of photopeak detection. A simple example will be examined using two methods of summing corrections; the simple mathematical correction suggested by G. Gilmore in \textit{Practical Gamma Ray Spectroscopy}, and a probabilistic examination using terms found in T. M. Semkow's correction method in \textit{Coincidence summing in $\gamma$-ray spectroscopy}.

For the purposes of this example, we will assume that the conversion coefficients for all gamma rays are 0 (\textbf{$\alpha$} = $0$). Table \ref{probabilities} (located in the ``Summary" subsection below) describes the terms necessary to describe gamma ray emission and detection in the following analysis, and elucidates the source of each term.

The simplest decay scheme that necessarily involves coincidence summing contains two excited energy states, as shown in Figure \ref{simple}. $\beta^{-}$ decay populates both excited states, producing emissions of  $\gamma_{1}$, $\gamma_{2}$, and  $\gamma_{3}$, and coincident photopeak detections of $\gamma_{1}$ and $\gamma_{2}$ can sum to produce a false count of $\gamma_{3}$. 

 %\begin{figure}[h!]
%	\centering
%	\includegraphics[width=6cm]{simple_scheme.png}
%	\caption{Simple decay scheme producing coincidence summing. $\gamma_{\ell,k}$ denotes the gamma emission from state $n_{\ell}$ to $n_{k}$, the feeding fractions (f$_{\ell}$) describe the ratio of initial electron populations of each energy state via $\beta^{-}$ decay, and the branching fractions ($x_{\ell,k}$) describe the probabilities of emissions for $\gamma$ rays from the $n_{\ell}$ energy state.}
%\label{simple}
%\end{figure}

In Figure \ref{simple}, $\gamma_{\ell,k}$ denotes a gamma emission produced by an electron relaxing from state $n_{\ell}$ to $n_{k}$. The initial population of each energy state via $\beta^{-}$ decay is described using "feeding fractions", labeled f$_{\ell}$. The feeding fractions describe the ratio of electrons populating each excited state, and sum of the feeding fractions is normalized to 1. Since population of the ground state does not produce gamma rays, it does not affect gamma ray detection and is set to 0. Once an excited energy state is populated, the probability of each subsequent emission is described using branching ratios, labeled $x_{\ell,k}$ in Figure \ref{simple}. The sum of all branching ratios for each $\ell^{th}$ state is equal to 1.

Since summing out reduces the number of detections of a full energy gamma emission and summing in increases the number of counts, the measured peak intensity (total number of counts) for a particular $\gamma$ ray photopeak is

\begin{equation}
\textbf{S}_{\ell,k} = \textbf{S}_{\ell,k}^{0} - \textbf{S}_{\ell,k}^{-} + \textbf{S}_{\ell,k}^{+} ,
\label{S}
\end{equation}
where \textbf{S}$_{\ell,k}^{0}$ is the intensity expected without summing, \textbf{S}$_{\ell,k}^{-}$ is the contribution due to summing out, and \textbf{S}$_{\ell,k}^{+}$ is the contribution due to summing in. 

\subsection{No Summing}

The intensity of a measured energy photopeak without any summing (\textbf{S}$_{\ell,k}^{0}$) depends on the probability of population of the $\ell^{th}$ energy state ($p_{\ell,k}$), and the probability of $\gamma_{\ell,k}$ emission ($\varepsilon_{\ell,k}$). Using the example in Figure \ref{simple}, the anticipated peak intensity of $\gamma_{1,0}$ can be described simply as a product of these quantities:

\begin{equation}
\textbf{S}_{1,0}^{0}  = p_{1,0} * \varepsilon_{1,0}.
\label{s1}
\end{equation} 

 \begin{figure}[h!]
	\centering
	\includegraphics[width=6cm]{paths10.png}
	\caption{Simple decay scheme highlighting the two decay paths that populate the 1$^{st}$ energy state. The red path is a result of direct $\beta^{-}$ decay feeding, and the green path involves $\beta^{-}$ population of the 2$^{nd}$ energy state and subsequent emission of $\gamma_{2,1}$.}
\label{paths}
\end{figure}

The probability of decay producing $\gamma_{1,0}$ can be expanded by describing all possible decay paths that populate the 1$^{st}$ energy state. Figure \ref{paths} shows the two decay paths: one path is by direct population via $\beta^{-}$ decay ($f_{1}$), and another is by population of the 2$^{nd}$ energy state ($f_{2}$) followed by a $\gamma_{2,1}$ emission (x$_{2,1}$). Substituting $p_{1,0}$ for $(f_{1} + f_{2} x_{2,1})$ in equation \ref{s1}, the intensity of $\gamma_{1,0}$ then becomes

\begin{equation}
\textbf{S}_{1,0}^{0} = (f_{1} + f_{2} \text{x}_{2,1}) * \varepsilon_{1,0}.
\label{s2}
\end{equation} 

To find the probability of $\gamma_{1,0}$ detection, the probability of $\gamma_{1,0}$ emission from the 1$^{st}$ energy state (x$_{1,0}$) is combined with the detection efficiency of at the energy of $\gamma_{1,0}$'s photopeak (or the detector's ``peak efficiency", $\epsilon^{p}_{1,0}$). Substituting $\varepsilon_{1,0}$ for $x_{1,0} \epsilon^{p}_{1,0}$ in equation \ref{s2},

\begin{equation}
\textbf{S}_{1,0}^{0} = (f_{1} + f_{2} \text{x}_{2,1}) * \text{x}_{1,0} \epsilon^{p}_{1,0}.
\label{s3}
\end{equation} 
Similarly, the relative intensities of $\gamma_{2,0}$ and $\gamma_{2,1}$ without summing are 

\begin{align}
&\textbf{S}_{2,0}^{0} = f_{2} * \text{x}_{2,0} \epsilon^{p}_{2,0}
&\textbf{S}_{2,1}^{0} = f_{2} * \text{x}_{2,1} \epsilon^{p}_{2,1}.
\label{probs}
\end{align}

\subsection{Summing In}

In the simple example above, only the largest peak, $\gamma_{2,0}$, will experience any significant summing in. Thus, 

\begin{equation}
\textbf{S}_{1,0}^{+} = \textbf{S}_{2,1}^{+} = 0
\label{s+}
\end{equation}

To find the counts of $\gamma_{2,0}$ caused by summing in, the probability of nearly simultaneous emissions of $\gamma_{2,1}$ and $\gamma_{1,0}$ needs to be determined. Since the lifetimes of each excited state are very short, it is expected that most coincident emissions will occur after a cascade of low-energy emissions that sum to a higher energy emission. For a simple system, this occurs when a $\gamma_{2,1}$ emission is immediately followed by $\gamma_{1,0}$, as shown in Figure \ref{sum_in}. 

 \begin{figure}[h!]
	\centering
	\includegraphics[width=6cm]{summing_in.png}
	\caption{Simple decay scheme highlighting the path of decay producing nearly simultaneous detection of $\gamma_{2,1}$ and $\gamma_{1,0}$. This involves $\beta^{-}$ population of the 2$^{nd}$ energy state and subsequent emission of $\gamma_{2,1}$ (green), followed by the emission of $\gamma_{1,0}$ (red).}
\label{sum_in}
\end{figure}

The probability of detecting the full energies of the summed emissions depends on three factors: the probability of $\gamma_{2,1}$ decay, the probability of $\gamma_{2,1}$ photopeak detection, and the probability of the following $\gamma_{1,0}$ photopeak detection:

\begin{equation}
\textbf{S}_{2,0}^{+} = p_{2,1} \varepsilon_{2,1}*\varepsilon_{1,0}
\label{s+}
\end{equation}
Expanding equation \ref{s+} as before, the counts of $\gamma_{2,0}$ attributed to summing in becomes

\begin{equation}
\textbf{S}_{2,0}^{+} = f_{2} \text{x}_{2,1} \epsilon^{p}_{2,1} * \text{x}_{1,0} \epsilon^{p}_{1,0}.
\label{s+2}
\end{equation}

Since coincidence summing only affects gamma rays hitting the same detector, the probability of detecting each subsequent $\gamma$ ray is divided by the number of detectors spanning the array. When using N individual detectors, equation \ref{s+2} becomes

\begin{equation}
\textbf{S}_{2,0}^{+} = \frac{1}{N} f_{2} x_{2,1} \epsilon^{p}_{2,1} * x_{1,0} \epsilon^{p}_{1,0}.
\label{s+3}
\end{equation}

\subsection{Summing Out}
%Very few (if any) partial gamma emissions will be coincident with the highest energy $\gamma$ emission in a spectrum, so for the simple system, 

%\begin{equation}
%\textbf{S}_{2,0}^{-} = 0
%\end{equation}

As previously described, summing out is caused by the simultaneous detection of a full energy $\gamma$ peak and \textit{any} an additional $\gamma$ ray energy. 

In the simple decay scheme shown in Figure \ref{sum_out}, $\gamma_{2,0}$ and $\gamma_{1,0}$ cannot be produced at the same time, so $\gamma_{2,0}$ detection will not contribute to summing out. To determine the degree of summing out for $\gamma_{1,0}$, the probability of photopeak detection of $\gamma_{1,0}$ is multiplied by the probability of $\gamma_{2,1}$ emission and the probability of $\gamma_{2,1}$ detection anywhere in the spectrum ($\varepsilon^{T}_{2,1}$):

\begin{equation}
\textbf{S}_{1,0}^{-} = \varepsilon_{1,0} * p_{2,1} * \varepsilon^{T}_{2,1} 
\label{s-}
\end{equation}

 \begin{figure}[h!]
	\centering
	\includegraphics[width=6cm]{summing_out.png}
	\caption{Simple decay scheme highlighting the path of decay producing nearly simultaneous detection of $\gamma_{2,1}$ and $\gamma_{1,0}$. This involves $\beta^{-}$ population of the 2$^{nd}$ energy state and subsequent emission of $\gamma_{2,1}$ (green), followed by the emission of $\gamma_{1,0}$ (red).}
\label{sum_out}
\end{figure}

The probability of $\gamma_{2,1}$ detection at some energy level is determined by its probability of emission ($x_{2,1}$) and the likelihood that some part of the $\gamma_{2,1}$ energy will be counted by the detector (referred to as the ``total efficiency" of $\gamma_{2,1}$ detection, $\epsilon^{t}_{2,1}$). The probabilities of $\gamma_{1,0}$ detection and $\gamma_{2,1}$ production ($p_{2,1}$) are rewritten as before, using Figure \ref{sum_out} to determine the decay path producing $\gamma_{2,1}$. Substituting this information into equation \ref{s-},

\begin{equation}
\textbf{S}_{1,0}^{-} = x_{1,0} \epsilon^{p}_{1,0} * f_{2} * x_{2,1} \epsilon^{t}_{2,1} 
\label{s-3}
\end{equation}

Similarly, the counts of $\gamma_{2,1}$ lost due to summing out depend on the probability of $\gamma_{2,1}$ photopeak detection ($x_{2,1} \epsilon^{p}_{2,1}$), the probability of $\gamma_{1,0}$ production ($f_{2} x_{2,1} + f_{1}$), and the probability of total $\gamma_{1,0}$ detection ($x_{1,0} \epsilon^{t}_{1,0}$):

\begin{equation}
\textbf{S}_{2,1}^{-} = x_{2,1} \epsilon^{p}_{2,1} * (f_{2} x_{2,1} + f_{1}) * x_{1,0} \epsilon^{t}_{1,0} 
\label{s-4}
\end{equation}

As described in the previous section, the probability of detection for each coincident $\gamma$ ray must be divided by the number of detectors used, so to describe the total contributions due to summing out, equations \ref{s-3} and \ref{s-4} become

\begin{align}
\textbf{S}_{1,0}^{-} &= \frac{1}{N} x_{1,0} \epsilon^{p}_{1,0} * f_{2} * x_{2,1} \epsilon^{t}_{2,1} \\
\textbf{S}_{2,1}^{-} &= \frac{1}{N} x_{2,1} \epsilon^{p}_{2,1} * (f_{2} x_{2,1} + f_{1}) * x_{1,0} \epsilon^{t}_{1,0}. 
\label{sumout}
\end{align}

\subsection{Summary}
First, the terms described in the sections above are aggregated and defined in Table \ref{probabilities} to clarify the relationships between the basic probabilistic variables used in Gilmore's text and the expanded terms used in Semkow's text to describe these variables. 

\begin{table}[h!]
\caption[Definitions of terms]{Definitions of terms logically, as described in Gilmore, and as described in Semkow. Items listed as "--" are not explicitly described in the text.} 
\begin{center}
\begin{tabular}{| p{3 cm} | p{4.5 cm} | l | l |}
\hline
\textbf{Element} & \textbf{Definition} & \textbf{Gilmore variable} & \textbf{Semkow variable} \\ 
\hline
\hline
Feeding fraction & Initial population of $\ell^{th}$ state by $\beta^{-}$ decay (normalized to 1) &--& f$_{\ell}$ \\ 
\hline
Branching ratio & Probability of decay to $k^{th}$ state from $\ell^{th}$ state (normalized to 1) $\ell$ & -- & x$_{\ell,k}$ \\
\hline
Peak efficiency & Efficiency of $\gamma$ detection at its photopeak. & -- & $\epsilon^{p}_{\ell,k}$ \\
\hline
Total efficiency & Efficiency of $\gamma$ detection at any energy (including background detection of partial $\gamma$ energy deposits) & -- & $\epsilon^{t}_{\ell,k}$ \\
\hline
Emission probability	& Likelihood that $\gamma_{\ell,k}$ emission will occur & p$_{\ell,k}$ & $\Sigma$ f$_{\ell,k}$x$_{\ell,k}$  \\
\hline
Peak probability & Probability of $\gamma$ detection at expected photopeak & $\varepsilon_{\ell,k}$ & x$_{\ell,k}$$\epsilon^{p}_{\ell,k}$\\
\hline
Detection probability & Probability of $\gamma$ detection at some energy level & $\varepsilon^{T}_{\ell,k}$ & x$_{\ell,k}$$\epsilon^{t}_{\ell,k}$\\
\hline
Conversion coefficient & Likelihood of conversion from gamma ray into other particle in detector & 0 & $\alpha_{\ell,k}$ \\
\hline 
\end{tabular}
\end{center}
\label{probabilities}
\end{table}

Then, the total measured counts expected for each $\gamma$ emission in the simple system are described by multiplying the total probabilities for each emission (using equation \ref{S}) by the source disintegration rate, $R$. Thus, the total counts expected for each $\gamma$ emission in the simple example are

\begin{align}
\textbf{S}_{1,0} &= R \left((f_{1} + f_{2} x_{2,1}) x_{1,0} \epsilon^{p}_{1,0} - \frac{1}{N} x_{1,0} \epsilon^{p}_{1,0} * f_{2} * x_{2,1} \epsilon^{t}_{2,1}\right) \\
\label{total1}
\textbf{S}_{2,1} &= R \left(f_{2} x_{2,1} \epsilon^{p}_{2,1} - \frac{1}{N} x_{2,1} \epsilon^{p}_{2,1} * (f_{2} x_{2,1} + f_{1}) * x_{1,0} \epsilon^{t}_{1,0}\right) \\
\textbf{S}_{2,0} &= R \left(f_{2} x_{2,0} \epsilon^{p}_{2,0} + \frac{1}{N} f_{2} x_{2,1} \epsilon^{p}_{2,1} * x_{1,0} \epsilon^{p}_{1,0}\right).
\label{total}
\end{align}


\section{Comparison of Mathematical Correction Methods}

To verify that program written to execute Semkow's correction method produces reasonable results for simple systems, the equations describing the total counts of gamma rays from the simple example (equations  \ref{total1}-\ref{total}) are used to calculate the expected photopeak intensities for a particular source and detector array, and the results are compared to the output of the program. 

 \begin{figure}[h!]
	\centering
	\includegraphics[width=14cm]{simplified_co.png}
	\caption{a) Full decay scheme for Co-60. b) Simplified Co-60 decay scheme achieved by removing the 2158.610 keV state.}
\label{simple_co}
\end{figure}

To compare these two methods, data for Co-60 gamma emission were modified to simplify the Co-60 decay scheme \cite{nuclide-co60}. As shown in Figure \ref{simple_co}, the three-level decay scheme was modified by removing the excited state that is least likely to be populated (n=2; E = 2158.610 keV). The data for the remaining gamma emissions are in Table \ref{simpledata}, alongside parameters of the GRIFFIN TRIUMF HPGe detection array.

\begin{table}[h!]
\caption[Simplified Data]{Simplified Co-60 scheme data and detector array data from the GRIFFIN TRIUMF detector used for comparison calculations \cite{GRIFFIN}.} 
\begin{center}
\begin{tabular}{| l | l |}
\hline
Simplified Co-60 data: & Detector array data: \\
\hline
\hline
$f_{2}$ = 0.9988 & $N$ = 64 \\
$f_{1}$ = 0.0012 & $\epsilon^{t}$ = 0.4 \\
$x_{2,0}$ = 2.00300447x10-8 & $\epsilon^{p}_{2,0}$ = 0.0676444 \\
$x_{2,1}$ = 0.999999980 & $\epsilon^{p}_{2,1}$ = 0.11635693 \\
$x_{1,0}$ = 1 & $\epsilon^{p}_{1,0}$ = 0.10756144 \\
$R$ = 100 &  \\
\hline
\end{tabular}
\end{center}
\label{simpledata}
\end{table}

Using the data from Table \ref{simpledata} and equations \ref{total1}-\ref{total}, the total counts of peaks for with coincidence summing were found to be 

\begin{align}
\textbf{S}_{1,0}(1332.508 keV) &= 10.69 \\
\textbf{S}_{2,1}(1173.240 keV) &=  11.55 \\
\textbf{S}_{2,0}(2505.748 keV) &= 0.01953,
\label{total}
\end{align}
and the counts without summing were

\begin{align}
\textbf{S}_{1,0}^{0}(1332.508 keV) &=  10.76 \\
\textbf{S}_{2,1}^{0}(1173.240 keV) &=  11.62 \\
\textbf{S}_{2,0}^{0}(2505.748 keV) &= 1.353 \times 10^{-7}.
\label{total}
\end{align}

 \begin{figure}[h!]
	\centering
	\includegraphics[width=14cm]{placeholder_plot.png}
	\caption{This is a placeholder! Current code cannot correctly calculate summing in!}Photopeak energies vs Counts cetected for simplified Co-60 decay scheme. Dark grey counts were found using Gilmore's method, and light grey counts were found using the program. Solid bars indicate counts without summing, and hatched bars indicate total measured counts.}
\label{plot1}
\end{figure}

\chapter{3. Setup for Correction Scheme}
Defined quantities input by the user:
peak efficiencies $\epsilon^{p}_{mn}$, total efficiency of detector $\epsilon^{t}_{mn}$, feeding radios for decay $f_{mn}$, branching ratios $x_{mn}$, and conversion coefficients $\alpha_{mn}$.


\chapter{3. Referential things}
\section{References, Labels, Custom Commands and Footnotes}
It is easy to refer to anything within your document using the \texttt{label} and \texttt{ref} tags.  Labels must be unique and shouldn't use any odd characters; generally sticking to letters and numbers (no spaces) should be fine. Put the label on whatever you want to refer to, and put the reference where you want the reference. \LaTeX\ will keep track of the chapter, section, and figure or table numbers for you. 

\subsection{References and Labels}
Sometimes you'd like to refer to a table or figure, e.g. you can see in Figure \ref{subd2} that you can rotate figures . Start by labeling your figure or table with the label command (\verb=\label{labelvariable}=) below the caption (see the chapter on graphics and tables for examples). Then when you would like to refer to the table or figure, use the ref command (\verb=\ref{labelvariable}=). Make sure your label variables are unique; you can't have two elements named ``default." Also, since the reference command only puts the figure or table number, you will have to put  ``Table" or ``Figure" as appropriate, as seen in the following examples:

 As I showed in Table \ref{inheritance} many factors can be assumed to follow from inheritance. Also see the Figure \ref{subd} for an illustration.
 
\subsection{Custom Commands}\label{commands}
Are you sick of writing the same complex equation or phrase over and over? 

The custom commands should be placed in the preamble, or at least prior to the first usage of the command. The structure of the \verb=\newcommand= consists of the name of the new command in curly braces, the number of arguments to be made in square brackets and then, inside a new set of curly braces, the command(s) that make up the new command. The whole thing is sandwiched inside a larger set of curly braces. 

% Note: you cannot use numbers in your commands!
\newcommand{\hydro}{H$_2$SO$_4$}

In other words, if you want to make a shorthand for H$_2$SO$_4$, which doesn't include an argument, you would write: \verb=\newcommand{\hydro}{H$_2$SO$_4$}= and then when you needed  to use the command you would type \verb=\hydro=. (sans verb and the equals sign brackets, if you're looking at the .tex version). For example: \hydro

\subsection{Footnotes and Endnotes}
	You might want to footnote something.\footnote{footnote text} Be sure to leave no spaces between the word immediately preceding the footnote command and the command itself. The footnote will be in a smaller font and placed appropriately. Endnotes work in much the same way. More information can be found about both on the CUS site.
	
\section{Bibliographies}
	Of course you will need to cite things, and you will probably accumulate an armful of sources. This is why BibTeX was created. For more information about BibTeX and bibliographies, see our CUS site (\url{web.reed.edu/cis/help/latex/index.html})\footnote{\cite{reedweb:2007}}. There are three pages on this topic: {\it bibtex} (which talks about using BibTeX, at \url{/latex/bibtex.html}), {\it bibtexstyles} (about how to find and use the bibliography style that best suits your needs, at \url{/latex/bibtexstyles.html}) and {\it bibman} (which covers how to make and maintain a bibliography by hand, without BibTeX, at at \url{/latex/bibman.html}). The last page will not be useful unless you have only a few sources. There used to be APA stuff here, but we don't need it since I've fixed this with my apa-good natbib style file.
	
\subsection{Tips for Bibliographies}
\begin{enumerate}
\item Like with thesis formatting, the sooner you start compiling your bibliography for something as large as thesis, the better. Typing in source after source is mind-numbing enough; do you really want to do it for hours on end in late April? Think of it as procrastination.
\item The cite key (a citation's label) needs to be unique from the other entries.
\item When you have more than one author or editor, you need to separate each author's name by the word ``and'' e.g.\\ \verb+Author = {Noble, Sam and Youngberg, Jessica},+.
\item Bibliographies made using BibTeX (whether manually or using a manager) accept LaTeX markup, so you can italicize and add symbols as necessary.
\item To force capitalization in an article title or where all lowercase is generally used, bracket the capital letter in curly braces.
\item You can add a Reed Thesis citation\footnote{\cite{noble:2002}} option. The best way to do this is to use the phdthesis type of citation, and use the optional ``type'' field to enter ``Reed thesis'' or ``Undergraduate thesis''. Here's a test of Chicago, showing the second cite in a row\footnote{\cite{noble:2002}} being different. Also the second time not in a row\footnote{\cite{reedweb:2007}} should be different. Of course in other styles they'll all look the same.
\end{enumerate}
\section{Anything else?}
If you'd like to see examples of other things in this template, please contact CUS (email cus@reed.edu) with your suggestions. We love to see people using \LaTeX\ for their theses, and are happy to help.


\chapter{Mathematics and Science}	
\section{Math}
	\TeX\ is the best way to typeset mathematics. Donald Knuth designed \TeX\ when he got frustrated at how long it was taking the typesetters to finish his book, which contained a lot of mathematics. 
	
	If you are doing a thesis that will involve lots of math, you will want to read the following section which has been commented out. If you're not going to use math, skip over this next big red section. (It's red in the .tex file but does not show up in the .pdf.)
%	
%% MATH and PHYSICS majors: Uncomment the following section	
%	$$\sum_{j=1}^n (\delta\theta_j)^2 \leq {{\beta_i^2}\over{\delta_i^2 + \rho_i^2}}
%\left[ 2\rho_i^2 + {\delta_i^2\beta_i^2\over{\delta_i^2 + \rho_i^2}} \right] \equiv \omega_i^2
%$$

%From Informational Dynamics, we have the following (Dave Braden):

%After {\it n} such encounters the posterior density for $\theta$ is

%$$
%\pi(\theta|X_1< y_1,\dots,X_n<y_n) \varpropto \pi(\theta) \prod_{i=1}^n\int_{-\infty}^{y_i}
%   \exp\left(-{(x-\theta)^2\over{2\sigma^2}}\right)\ dx
%$$

%

%Another equation:

%$$\det\left|\,\begin{matrix}%
%c_0&c_1\hfill&c_2\hfill&\ldots&c_n\hfill\cr
%c_1&c_2\hfill&c_3\hfill&\ldots&c_{n+1}\hfill\cr
%c_2&c_3\hfill&c_4\hfill&\ldots&c_{n+2}\hfill\cr
%\,\vdots\hfill&\,\vdots\hfill&
%  \,\vdots\hfill&&\,\vdots\hfill\cr
%c_n&c_{n+1}\hfill&c_{n+2}\hfill&\ldots&c_{2n}\hfill\cr
%\end{matrix}\right|>0$$

%
%Lapidus and Pindar, Numerical Solution of Partial Differential Equations in Science and
%Engineering.  Page 54

%$$
%\int_t\left\{\sum_{j=1}^3 T_j \left({d\phi_j\over dt}+k\phi_j\right)-kT_e\right\}w_i(t)\ dt=0,
%   \qquad\quad i=1,2,3. 
%$$

%L\&P  Galerkin method weighting functions.  Page 55

%$$
%\sum_{j=1}^3 T_j\int_0^1\left\{{d\phi_j\over dt} + k\phi_j\right\} \phi_i\ dt 
%   = \int_{0}^1k\,T_e\phi_idt, \qquad i=1,2,3 $$
%   
%Another L\&P (p145)

%$$
%\int_{-1}^1\!\int_{-1}^1\!\int_{-1}^1 f\big(\xi,\eta,\zeta\big) 
%   = \sum_{k=1}^n\sum_{j=1}^n\sum_{i=1}^n w_i w_j w_k f\big( \xi,\eta,\zeta\big).
%$$

%Another L\&P (p126)

%$$
%\int_{A_e} (\,\cdot\,) dx dy = \int_{-1}^1\!\int_{-1}^1 (\,\cdot\,) \det[J] d\xi d\eta.
%$$

\section{Chemistry 101: Symbols}
Chemical formulas will look best if they are not italicized. Get around math mode's automatic italicizing by using the argument \verb=$\mathrm{formula here}$=, with your formula inside the curly brackets.

So, $\mathrm{Fe_2^{2+}Cr_2O_4}$ is written \verb=$\mathrm{Fe_2^{2+}Cr_2O_4}$=\\
Exponent or Superscript: O$^{-}$\\
Subscript: CH$_{4}$\\

To stack numbers or letters as in $\mathrm{Fe_2^{2+}}$, the subscript is defined first, and then the superscript is defined.\\
Angstrom: {\AA}\\
Bullet: CuCl $\bullet$ 7H${_2}$O\\
Double Dagger: \ddag \/\\
Delta: $\Delta$\\
Reaction Arrows: $\longrightarrow$ or  $\xrightarrow{solution}$\\
Resonance Arrows: $\leftrightarrow$\\
Reversible Reaction Arrows: $\rightleftharpoons$ or $\xrightleftharpoons[ ]{solution}$ (the latter requires the chemarr package)\\


\subsection{Typesetting reactions}
You may wish to put your reaction in a figure environment, which means that LaTeX will place the reaction where it fits and you can have a figure legend if desired:
\begin{figure}[htbp]
\begin{center}
$\mathrm{C_6H_{12}O_6  + 6O_2} \longrightarrow \mathrm{6CO_2 + 6H_2O}$
\caption{Combustion of glucose}
\label{combustion of glucose}
\end{center}
\end{figure}

\subsection{Other examples of reactions}
$\mathrm{NH_4Cl_{(s)}} \rightleftharpoons \mathrm{NH_{3(g)}+HCl_{(g)}}$\\
$\mathrm{MeCH_2Br + Mg} \xrightarrow[below]{above} \mathrm{MeCH_2\bullet Mg \bullet Br}$

\section{Physics}

Many of the symbols you will need can be found on the math page (\url{http://web.reed.edu/cis/help/latex/math.html}) and the Comprehensive \LaTeX\ Symbol Guide (enclosed in this template download).  You may wish to create custom commands for commonly used symbols, phrases or equations, as described in Chapter \ref{commands}.

\section{Biology}
You will probably find the resources at \url{http://www.lecb.ncifcrf.gov/~toms/latex.html} helpful, particularly the links to bsts for various journals. You may also be interested in TeXShade for nucleotide typesetting (\url{http://homepages.uni-tuebingen.de/beitz/txe.html}).  Be sure to read the proceeding chapter on graphics and tables, and remember that the thesis template has versions of Ecology and Science bsts which support webpage citation formats. 

\chapter{Tables and Graphics}

\section{Tables}
	The following section contains examples of tables, most of which have been commented out for brevity. (They will show up in the .tex document in red, but not at all in the .pdf). For more help in constructing a table (or anything else in this document), please see the LaTeX pages on the CUS site. 

\begin{table}[htbp] % begins the table floating environment. This enables LaTeX to fit the table where it works best and lets you add a caption.
\caption[Correlation of Inheritance Factors between Parents and Child]{Correlation of Inheritance Factors between Parents and Child} 
% The words in square brackets of the caption command end up in the Table of Tables. The words in curly braces are the caption directly over the table.
\begin{center} 
% makes the table centered
\begin{tabular}{c c c c} 
% the tabular environment is used to make the table itself. The {c c c c} specify that the table will have four columns and they will all be center-aligned. You can make the cell contents left aligned by replacing the Cs with Ls or right aligned by using Rs instead. Add more letters for more columns, and pipes (the vertical line above the backslash) for vertical lines. Another useful type of column is the p{width} column, which forces text to wrap within whatever width you specify e.g. p{1in}. Text will wrap badly in narrow columns though, so beware.
\toprule % a horizontal line, slightly thicker than \hline, depends on the booktabs package
  Factors &  Correlation between Parents \& Child & Inherited \\ % the first row of the table. Separate columns with ampersands and end the line with two backslashes. An environment begun in one cell will not carry over to adjacent rows.
  \midrule % another horizontal line
	Education 				& -0.49 & Yes 	 \\ % another row
	Socio-Economic Status 	& 0.28 	& Slight \\
	Income 					& 0.08 	& No	 \\
	Family Size 			& 0.19 	& Slight \\
	Occupational Prestige 	& 0.21 	& Slight \\
\bottomrule % yet another horizontal line
\end{tabular}
\end{center}
\label{inheritance} % labels are useful when you have more than one table or figure in your document. See our online documentation for more on this.
\end{table}

	\clearpage 
%% \clearpage ends the page, and also dumps out all floats. 
%% Floats are things like tables and figures.

If you want to make a table that is longer than a page, you will want to use the longtable environment. Uncomment the table below to see an example, or see our online documentation.

%% An example of a long table, with headers that repeat on each subsequent page: Results from the summers of 1998 and 1999 work at Reed College done by Grace Brannigan, Robert Holiday and Lien Ngo in 1998 and Kate Brown and Christina Inman in 1999.

	\begin{longtable}{||c|c|c|c||}
	 	\caption[Chromium Hexacarbonyl Data Collected in 1998--1999]{Chromium Hexacarbonyl Data Collected in 1998--1999}\\ \hline
	    	  \multicolumn{4}{||c||}{Chromium Hexacarbonyl} \\\hline
		   State & Laser wavelength & Buffer gas & Ratio of $\frac{\textrm{Intensity
at vapor pressure}}{\textrm{Intensity at 240 Torr}}$ \\ \hline
		  \endfirsthead
		\hline     State & Laser wavelength & Buffer gas & Ratio of
$\frac{\textrm{Intensity at vapor pressure}}{\textrm{Intensity at 240 Torr}}$\\
\hline
		    \endhead

	    $z^{7}P^{\circ}_{4}$ & 266 nm & Argon & 1.5 \\\hline
	    $z^{7}P^{\circ}_{2}$ & 355 nm & Argon & 0.57 \\\hline
	    $y^{7}P^{\circ}_{3}$ & 266 nm & Argon & 1 \\\hline
	    $y^{7}P^{\circ}_{3}$ & 355 nm & Argon & 0.14 \\\hline
	    $y^{7}P^{\circ}_{2}$ & 355 nm & Argon & 0.14 \\\hline
	    $z^{5}P^{\circ}_{3}$ & 266 nm & Argon & 1.2 \\\hline
	    $z^{5}P^{\circ}_{3}$ & 355 nm & Argon & 0.04 \\\hline
	    $z^{5}P^{\circ}_{3}$ & 355 nm & Helium & 0.02 \\\hline
	    $z^{5}P^{\circ}_{2}$ & 355 nm & Argon & 0.07 \\\hline
	    $z^{5}P^{\circ}_{1}$ & 355 nm & Argon & 0.05 \\\hline
	    $y^{5}P^{\circ}_{3}$ & 355 nm & Argon & 0.05, 0.4 \\\hline
	    $y^{5}P^{\circ}_{3}$ & 355 nm & Helium & 0.25 \\\hline
	    $z^{5}F^{\circ}_{4}$ & 266 nm & Argon & 1.4 \\\hline
	    $z^{5}F^{\circ}_{4}$ & 355 nm & Argon & 0.29 \\\hline
	    $z^{5}F^{\circ}_{4}$ & 355 nm & Helium & 1.02 \\\hline
	    $z^{5}D^{\circ}_{4}$ & 355 nm & Argon & 0.3 \\\hline
	    $z^{5}D^{\circ}_{4}$ & 355 nm & Helium & 0.65 \\\hline
	    $y^{5}H^{\circ}_{7}$ & 266 nm & Argon & 0.17 \\\hline
	    $y^{5}H^{\circ}_{7}$ & 355 nm & Argon & 0.13 \\\hline
	    $y^{5}H^{\circ}_{7}$ & 355 nm & Helium & 0.11 \\\hline
	    $a^{5}D_{3}$ & 266 nm & Argon & 0.71 \\\hline
	    $a^{5}D_{2}$ & 266 nm & Argon & 0.77 \\\hline
	    $a^{5}D_{2}$ & 355 nm & Argon & 0.63 \\\hline
	    $a^{3}D_{3}$ & 355 nm & Argon & 0.05 \\\hline
	    $a^{5}S_{2}$ & 266 nm & Argon & 2 \\\hline
	    $a^{5}S_{2}$ & 355 nm & Argon & 1.5 \\\hline
	    $a^{5}G_{6}$ & 355 nm & Argon & 0.91 \\\hline
	    $a^{3}G_{4}$ & 355 nm & Argon & 0.08 \\\hline
	    $e^{7}D_{5}$ & 355 nm & Helium & 3.5 \\\hline
	    $e^{7}D_{3}$ & 355 nm & Helium & 3 \\\hline
	    $f^{7}D_{5}$ & 355 nm & Helium & 0.25 \\\hline
	    $f^{7}D_{5}$ & 355 nm & Argon & 0.25 \\\hline
	    $f^{7}D_{4}$ & 355 nm & Argon & 0.2 \\\hline
	    $f^{7}D_{4}$ & 355 nm & Helium & 0.3 \\\hline
	    \multicolumn{4}{||c||}{Propyl-ACT} \\\hline
%	    State & Laser wavelength & Buffer gas & Ratio of $\frac{\textrm{Intensity
%at vapor pressure}}{\textrm{Intensity at 240 Torr}}$\\ \hline
	    $z^{7}P^{\circ}_{4}$ & 355 nm & Argon & 1.5 \\\hline
	    $z^{7}P^{\circ}_{3}$ & 355 nm & Argon & 1.5 \\\hline
	    $z^{7}P^{\circ}_{2}$ & 355 nm & Argon & 1.25 \\\hline
	    $z^{7}F^{\circ}_{5}$ & 355 nm & Argon & 2.85 \\\hline
	    $y^{7}P^{\circ}_{4}$ & 355 nm & Argon & 0.07 \\\hline
	    $y^{7}P^{\circ}_{3}$ & 355 nm & Argon & 0.06 \\\hline
	    $z^{5}P^{\circ}_{3}$ & 355 nm & Argon & 0.12 \\\hline
	    $z^{5}P^{\circ}_{2}$ & 355 nm & Argon & 0.13 \\\hline
	    $z^{5}P^{\circ}_{1}$ & 355 nm & Argon & 0.14 \\\hline
	    \multicolumn{4}{||c||}{Methyl-ACT} \\\hline
%	    State & Laser wavelength & Buffer gas & Ratio of $\frac{\textrm{Intensity
% at vapor pressure}}{\textrm{Intensity at 240 Torr}}$\\ \hline
	    $z^{7}P^{\circ}_{4}$ & 355 nm & Argon & 1.6, 2.5 \\\hline
	    $z^{7}P^{\circ}_{4}$ & 355 nm & Helium & 3 \\\hline
	    $z^{7}P^{\circ}_{4}$ & 266 nm & Argon & 1.33 \\\hline
	    $z^{7}P^{\circ}_{3}$ & 355 nm & Argon & 1.5 \\\hline
	    $z^{7}P^{\circ}_{2}$ & 355 nm & Argon & 1.25, 1.3 \\\hline
	    $z^{7}F^{\circ}_{5}$ & 355 nm & Argon & 3 \\\hline
	    $y^{7}P^{\circ}_{4}$ & 355 nm & Argon & 0.07, 0.08 \\\hline
	    $y^{7}P^{\circ}_{4}$ & 355 nm & Helium & 0.2 \\\hline
	    $y^{7}P^{\circ}_{3}$ & 266 nm & Argon & 1.22 \\\hline
	    $y^{7}P^{\circ}_{3}$ & 355 nm & Argon & 0.08 \\\hline
	    $y^{7}P^{\circ}_{2}$ & 355 nm & Argon & 0.1 \\\hline
	    $z^{5}P^{\circ}_{3}$ & 266 nm & Argon & 0.67 \\\hline
	    $z^{5}P^{\circ}_{3}$ & 355 nm & Argon & 0.08, 0.17 \\\hline
	    $z^{5}P^{\circ}_{3}$ & 355 nm & Helium & 0.12 \\\hline
	    $z^{5}P^{\circ}_{2}$ & 355 nm & Argon & 0.13 \\\hline
	    $z^{5}P^{\circ}_{1}$ & 355 nm & Argon & 0.09 \\\hline
	    $y^{5}H^{\circ}_{7}$ & 355 nm & Argon & 0.06, 0.05 \\\hline
	    $a^{5}D_{3}$ & 266 nm & Argon & 2.5 \\\hline
	    $a^{5}D_{2}$ & 266 nm & Argon & 1.9 \\\hline
	    $a^{5}D_{2}$ & 355 nm & Argon & 1.17 \\\hline
	    $a^{5}S_{2}$ & 266 nm & Argon & 2.3 \\\hline
	    $a^{5}S_{2}$ & 355 nm & Argon & 1.11 \\\hline
	    $a^{5}G_{6}$ & 355 nm & Argon & 1.6 \\\hline
	    $e^{7}D_{5}$ & 355 nm & Argon & 1 \\\hline

		\end{longtable}

   
   \section{Figures}
   
	If your thesis has a lot of figures, \LaTeX\ might behave better for you than that other word processor.  One thing that may be annoying is the way it handles ``floats'' like tables and figures. \LaTeX\ will try to find the best place to put your object based on the text around it and until you're really, truly done writing you should just leave it where it lies.   There are some optional arguments to the figure and table environments to specify where you want it to appear; see the comments in the first figure.

	If you need a graphic or tabular material to be part of the text, you can just put it inline. If you need it to appear in the list of figures or tables, it should be placed in the floating environment. 
	
	To get a figure from StatView, JMP, SPSS or other statistics program into a figure, you can print to pdf or save the image as a jpg or png. Precisely how you will do this depends on the program: you may need to copy-paste figures into Photoshop or other graphic program, then save in the appropriate format.
	
	Below we have put a few examples of figures. For more help using graphics and the float environment, see our online documentation.
	
	And this is how you add a figure with a graphic:
	\begin{figure}[h]
	% the options are h = here, t = top, b = bottom, p = page of figures.
	% you can add an exclamation mark to make it try harder, and multiple
	% options if you have an order of preference, e.g.
	% \begin{figure}[h!tbp]
	   
	       \centering
	    % DO NOT ADD A FILENAME EXTENSION TO THE GRAPHIC FILE
	    \includegraphics{subdivision}
	     \caption{A Figure}
	 \label{subd}
	\end{figure}

\clearpage %% starts a new page and stops trying to place floats such as tables and figures

\section{More Figure Stuff}
You can also scale and rotate figures.
 	\begin{figure}[h!]
	   
	       \centering
	    % DO NOT ADD A FILENAME EXTENSION TO THE GRAPHIC FILE
	    \includegraphics[scale=0.5,angle=180]{subdivision}
	    % if your figure shows up not where you want it, it may just be too big to fit. You can use the scale argument to shrink it, e.g. scale=0.85 is 85 percent of the original size. 
	     \caption{A Smaller Figure, Flipped Upside Down}
	 \label{subd2}
	\end{figure}

\section{Even More Figure Stuff}
With some clever work you can crop a figure, which is handy if (for instance) your EPS or PDF is a little graphic on a whole sheet of paper. The viewport arguments are the lower-left and upper-right coordinates for the area you want to crop.

 	\begin{figure}[h!]
	    	       \centering
	    % DO NOT ADD A FILENAME EXTENSION TO THE GRAPHIC FILE
	   \includegraphics[clip=true, viewport=.0in .0in 1in 1in]{subdivision}
	    \caption{A Cropped Figure}
	 \label{subd3}
	\end{figure}
	
      \subsection{Common Modifications}
      The following figure features the more popular changes thesis students want to their figures. This information is also on the web at \url{web.reed.edu/cis/help/latex/graphics.html}.
    %\renewcommand{\thefigure}{0.\arabic{figure}} 	% Renumbers the figure to the type 0.x
    %\addtocounter{figure}{4} 						% starts the figure numbering at 4
    \begin{figure}[htbp]
    \begin{center}
   \includegraphics[scale=0.5]{subdivision}
    \caption[Subdivision of arc segments]{\footnotesize{Subdivision of arc segments. You can see that $ p_3 = p_6^\prime$.}} %the special ToC caption is in square brackets. The \footnotesize makes the figure caption smaller
    \label{barplot}
    \end{center}
    \end{figure} 

\chapter*{Conclusion}
         \addcontentsline{toc}{chapter}{Conclusion}
	\chaptermark{Conclusion}
	\markboth{Conclusion}{Conclusion}
	\setcounter{chapter}{4}
	\setcounter{section}{0}
	
Here's a conclusion, demonstrating the use of all that manual incrementing and table of contents adding that has to happen if you use the starred form of the chapter command. The deal is, the chapter command in \LaTeX\ does a lot of things: it increments the chapter counter, it resets the section counter to zero, it puts the name of the chapter into the table of contents and the running headers, and probably some other stuff. 

So, if you remove all that stuff because you don't like it to say ``Chapter 4: Conclusion'', then you have to manually add all the things \LaTeX\ would normally do for you. Maybe someday we'll write a new chapter macro that doesn't add ``Chapter X'' to the beginning of every chapter title.



%If you feel it necessary to include an appendix, it goes here.
    \appendix
      \chapter{The First Appendix}

  \backmatter % backmatter makes the index and bibliography appear properly in the t.o.c...

% if you're using bibtex, the next line forces every entry in the bibtex file to be included
% in your bibliography, regardless of whether or not you've cited it in the thesis.
    \nocite{*}

% Rename my bibliography to be called "Works Cited" and not "References" or ``Bibliography''
% \renewcommand{\bibname}{Works Cited}

%    \bibliographystyle{bsts/mla-good} % there are a variety of styles available; 
%  \bibliographystyle{plainnat}
% replace ``plainnat'' with the style of choice. You can refer to files in the bsts or APA 
% subfolder, e.g. 
 \bibliographystyle{APA/apa-good}  % or
 \bibliography{thesis}
 % Comment the above two lines and uncomment the next line to use biblatex-chicago.
 %\printbibliography[heading=bibintoc]

% Finally, an index would go here... but it is also optional.
\end{document}
